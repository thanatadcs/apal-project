% This is a template for doing homework assignments in LaTeX

\documentclass{article} % This command is used to set the type of document you are working on such as an article, book, or presenation

\usepackage[margin=1in]{geometry} % This package allows the editing of the page layout
\usepackage{amsmath}  % This package allows the use of a large range of mathematical formula, commands, and symbols
\usepackage{graphicx}  % This package allows the importing of images
\usepackage{amsfonts}
\usepackage{enumitem}
\usepackage{listings}
\usepackage{hyperref}

% \setlength\parindent{0pt}

\newcommand{\question}[2][]{\begin{flushleft}
        \textbf{Question #1}: \textit{#2}

\end{flushleft}}
\newcommand{\sol}{\textbf{Solution}:} %Use if you want a boldface solution line
\newcommand{\maketitletwo}[2][]{\begin{center}
        \Large{\textbf{Project #1}
            
            ICCS315: Applied Algorithms} % Name of course here
        \vspace{5pt}
        
        \normalsize{Thanatad Anukoolprasert  % Your name here
        
        \today}        % Change to due date if preferred
        \vspace{15pt}
        
\end{center}}
\begin{document}
    \maketitletwo[]  % Optional argument is assignment number
    %Keep a blank space between maketitletwo and \question[1]
    
    \section*{Objective}
    The objective of this project is to benchmark two different type of data structures and compare different
    implementation of each type as well as their theoretical running time.
    The first type is resizable array where the following data structures is of interest
    \begin{itemize}
        \item Hashed Array Tree (Sitarski, 1996)
        \item Brodnik's non-superblock version (Brodnik et al., 1999)
        \item Brodnik's superblock version (Brodnik et al., 1999)
    \end{itemize}
    
    The second type is hash table with 3 different collision resolution schemes, namely
    \begin{itemize}
        \item Chaining
        \item Open addressing
        \item Cuckoo hashing
    \end{itemize}
    
    For a lack of a better name, we are going to called both the data structures from \href{ https://sedgewick.io/wp-content/themes/sedgewick/papers/1999Optimal.pdf}{ Brodnik et al 1999}, Hashed Array Tree (HAT).
    Not only that we are going to called Brodnik's HAT with non-superblock version \emph{Brodnik's HAT A} and the superblock version \emph{Brodnik's HAT B}.

    The dimensions of interest are the following
    \begin{itemize}
        \item Append latency
        \item Access latency
        \item Scan throughput
        \item Overall throughput
    \end{itemize}

    \section*{Method}
    We are going the implement the data structures above and benchmark them. We are going the measure running time in cycles using
    \emph{rdtsc} assembly instruction in x86 so that we are able to time quick operation like append with accurary.

    \subsection*{Remark on implementation}
    The implementation of \emph{get} function of Brodnik's HAT B is different from \emph{locate} function in the original paper.
    Namely, I cannot implement part of \emph{locate} function since there might be something wrong with the pseudocode in the paper,
    specifically \emph{$p=2^k - 1$} mentioned in the paper is does not represent the number of data blocks before the $k$-th superblock,
    but it represent the number of elements before the $k$-th superblock instead. For this reason I have to implement my own way to get the number of
    data blocks before the $k$-th superblock. More information can be found in brodnik-hat-b.cpp.

    \section*{Results}
    \subsection*{Hashed Array Tree}
    \subsection*{Hash table}

    \section*{Discussion}
    \subsection*{Limitation}

    \section*{Conclusion}

\end{document}
